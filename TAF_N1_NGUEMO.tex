\documentclass[12pt,a4paper]{article}
\usepackage[utf8]{inputenc}
\usepackage[T1]{fontenc}
\usepackage[french]{babel}
\usepackage{graphicx}
\usepackage{xcolor}      % pour les couleurs
\usepackage{geometry}
\usepackage{tikz}
\usetikzlibrary{calc}
\usepackage[french]{babel}
\usepackage{amsmath,amssymb,amsfonts}
\geometry{margin=2.5 cm}

\begin{document}
	\begin{titlepage}
		\begin{tikzpicture}[remember picture,overlay]
			\draw[line width=2pt,blue]
			($(current page.north west) + (1cm,-1cm)$) rectangle
			($(current page.south east) + (-1cm,1cm)$);
		\end{tikzpicture}
		
		
		% --- Logos et textes bilingues ---
		\noindent
		\begin{minipage}{0.45\textwidth}
			\centering
			{\small\textbf{REPUBLIQUE DU CAMEROUN}} \\  
			******\\
			{Paix – Travail – Patrie} \\
			******\\
			{\small\textbf {MINISTÈRE DE L’ENSEIGNEMENT SUPÉRIEUR}}\\
			******\\ 
			Université de Yaoundé I \\ 
			*****\\
			{\small\textbf{ÉCOLE NATIONALE SUPÉRIEURE POLYTECHNIQUE}} \\ 
			******\\
			Département de Génie Informatique \\
			******
			
			
		\end{minipage}
		\hfill
		\begin{minipage}{0.45\textwidth}
			\centering
			{\small\textbf{REPUBLIC OF CAMEROON}} \\ 
			******\\
			Peace – Work – Fatherland \\
			*******\\
			{\small\textbf{MINISTRY OF HIGHER EDUCATION}} \\
			******\\ 
			University of Yaoundé I \\ 
			******\\
			{\small\textbf{NATIONAL ADVANCED SCHOOL OF ENGINEERING}} \\
			****** \\
			Computer Engineering Department \\
			******
			
		\end{minipage}
		\begin{center}
			\includegraphics[width=3 cm]{poooo.png} % logo centré
		\end{center}
		
		
		\vspace{1cm}
		
		% --- Encadré principal  ---
		\begin{center}
		
			\colorbox{blue!70}{\parbox{1\textwidth}{\centering
					\textbf{\color{white}{SEC 4031 \quad INTRODUCTION AUX TECHNIQUES D'INVESTTGATION NUMERIQUE }}
					}}
			\vspace{0.5cm}
			
			\colorbox{gray!30}{\parbox{0.9\textwidth}{\centering
					\textbf{DEVOIR: Philosophie et Fondements de l'Investigation Numérique }
			}}
		\end{center}
		
		\vspace{1cm}
		
		% --- Infos filière ---
		\noindent
		\textbf{Filière :} HUMANITÉS NUMÉRIQUES \\[0.5cm]
		\textbf{Niveau :} IV
		
		\vspace{1cm}
		
		% --- Auteurs ---
		\noindent
		\textbf{Rédigé par :} \\[0.5cm]
		\quad          NGUEMO VOUFO AURELLE SANDRA \hfill 22P067\hfill CIN4
		
		\vspace{2cm}
		
		% --- Superviseur ---
		\begin{center}
			Sous la supervision de : \\
			\textbf{\hfill Mr. MINKA}\\[1 cm]
			\textbf{ANNEE ACADEMIQUE:2025-2026}
		\end{center}
		
	\end{titlepage}
	\section*{Partie1: Fondements Philosophiques et Épistémologiques }
	\subsection*{1. Analyse Critique du Paradoxe de la Transparence}
	* Le philosophe coréen Byung-Chul Han, dans la Société de la transparence, met en évidence un paradoxe central de notre ère numérique : l’exigence croissante de transparence, perçue comme une valeur démocratique et un gage de confiance, engendre en réalité une perte de liberté et une érosion de la vie privée.
	 Dans la sociétés, la transparence est présentée comme un idéal moral et politique. Elle garantit la surveillance des gouvernants, favorise la circulation de l’information et renforce la responsabilité. Pourtant, selon Han, cette transparence absolue se transforme en tyrannie : l’individu est exposé en permanence, scruté, mesuré, noté.Un exemple concret illustre ce paradoxe : les programmes de surveillance gouvernementale, Présentés comme outils de protection contre le terrorisme, ils légitiment une collecte massive de données personnelles. La transparence de l’État vis-à-vis de ses citoyens est alors inversée : ce sont les citoyens qui deviennent totalement transparents pour l’État. La promesse démocratique est ainsi trahie.\\[0.5cm]
	 * Ce paradoxe se retrouve dans l’investigation numérique. L’enquêteur doit arbitrer entre deux exigences contradictoires : établir la vérité (qui suppose transparence des données, accès aux traces numériques) et respecter la vie privée des individus. Trop de transparence détruit la confiance, trop de secret empêche la justice.\\[0.5cm]
	 * Une piste peut être trouvée dans l’éthique kantienne. Kant propose comme principe fondamental l’impératif catégorique : « Agis uniquement d’après la maxime grâce à laquelle tu peux vouloir en même temps qu’elle devienne une loi universelle ». Transposé au numérique, cela signifie : collecter et traiter les données uniquement selon des règles que tout citoyen pourrait accepter comme universelles. Par exemple, accéder aux données privées uniquement lorsqu’il existe une menace avérée et proportionnée.\\[0.1cm]
	 \subsection*{2.Transformation Ontologique du Numérique }
	 * Martin Heidegger analyse la technique comme un mode de dévoilement de l’être (Gestell). L’homme moderne ne se définit plus seulement par sa présence physique mais par son rapport technique au monde. À l’ère numérique, cette analyse prend une dimension nouvelle : l’individu existe désormais à travers un double numérique, constitué de ses données et traces.\\[0.5cm]
	 * Ce phénomène peut être décrit comme un « être-par-la-trace ». Un profil sur un réseau social comme Facebook est une projection de l’identité. Il ne s’agit pas d’une simple représentation mais d’une véritable existence, la personne interagit, communique, crée des effets sociaux à travers cette présence numérique. L’ontologie elle-même est modifiée : l’homme devient hybride( identité physique + identité numérique)\\[0.5cm]
	 * la preuve légale doit intégrer ces traces numériques comme manifestations d’existence, mais en gardant une approche critique (volatilité, falsification possible).Ainsi, la justice doit élargir son ontologie : reconnaître que l’être humain existe aussi dans ses traces numériques, tout en mettant en place des garanties techniques et juridiques pour assurer leur authenticité. Le numérique transforme non seulement la preuve, mais la définition même de l’existence légale.
	 \section*{Partie2: Mathématiques de l’Investigation}
	 \subsection*{3: Calcul d'entropie de Shannon appliquée}
	 
	 L'entropie de Shannon permet de mesurer l'incertitude contenue dans un fichier. Elle se calcule par :
	 \[
	 H(X) \;=\; - \sum_{x} p(x)\,\log_2 p(x)
	 \]
	 où \(p(x)\) est la probabilité d'apparition d'un symbole \(x\).
	 
	 * Interprétation pratique
	 \begin{itemize}
	 	 \textbf{Document texte (français naturel)} : entropie approximative \(H \approx 1{.}5\) bits par caractère. Les lettres fréquentes (par ex. « e », « a ») réduisent l'incertitude.
	 	 
	 	 \textbf{Image JPEG} : entropie approximative \(H \approx 7{.}2\) bits par octet. La compression supprime des redondances et augmente l'imprévisibilité statistique des octets.
	 	\textbf{Fichier chiffré AES} : entropie approximative \(H \approx 7{.}9\) bits par octet, proche de l'entropie maximale (uniformité).
	 	
	 \end{itemize}
	 
	 * Seuil de détection du chiffrement
	 Sur la base de ces valeurs, un seuil simple de détection automatique peut être posé :
	 \[
	 \text{Seuil}_{\text{chiffrement}} = 7{.}5 \text{ bits/octet}.
	 \]
	 Tout fichier ayant \(H > 7{.}5\) peut être considéré comme \emph{probablement chiffré}, ce qui déclenche une inspection plus approfondie par des experts.
	 
	 \subsection*{4: Théorie des graphes en investigation criminelle}
	 
	 Soit un graphe orienté ou non orienté \(G = (V,E)\) représentant un réseau de communications :
	 \begin{itemize}
	 	\item \(V\) : ensemble de personnes (abonnés),
	 	\item \(E\) : arêtes représentant les communications (appels, SMS, messages).
	 \end{itemize}
	 
	 * Métriques usuelles
	 \begin{description}
	 	\item[Degré] \( \deg(v)\) : nombre de connexions directes. Identifie les individus les plus actifs.
	 	\item[Centralité d'intermédiarité (betweenness)] : mesure le nombre de plus courts chemins passant par un nœud ; révèle les \emph{passeurs} ou brokers.
	 	\item[Centralité de proximité (closeness)] : inverse de la distance moyenne aux autres nœuds ; identifie les nœuds « centraux » en termes d'accessibilité.
	 \end{description}
	 
	 L'algorithme de Freeman (centralité de degré normalisée) et d'autres algorithmes (PageRank, centralité d'intermédiarité de Brandes) permettent d'identifier des cibles prioritaires. Dans un réseau criminel, le chef n'est pas nécessairement le plus connecté ; il peut être un nœud avec forte intermédiarité et faible visibilité publique.
	 
	 * Application
	 
	 L'analyse des communications téléphoniques peut révéler des coordinateurs discrets (faible degré apparente mais forte intermédiarité). La visualisation (avec tailles et couleurs proportionnelles aux mesures de centralité) facilite l'orientation des enquêtes.
	 
	 \subsection*{5: Modélisation de l'effet papillon en forensique}
	 
	 Considérons un système de logs comprenant \(N=1000\) événements corrélés. La modification d'un seul timestamp (par ex. déplacement aléatoire de \(\pm 30\) secondes) peut avoir un effet en cascade sur plusieurs événements corrélés. Une estimation heuristique donnée dans le chapitre est :
	 \[
	 \lceil \log_2(1000) \rceil \;=\; 10,
	 \]
	 ce qui signifie qu'une altération peut perturber la corrélation d'environ 10 événements connexes.
	 
	 \subsection*{Modèle dynamique}
	 On modélise l'évolution d'une erreur \(\delta(t)\) par :
	 \[
	 \delta(t) \approx \delta(0)\, e^{\lambda t},
	 \]
	 où \(\lambda\) est l'exposant de Lyapunov effectif du système. Plus \(\lambda\) est grand, plus le système est sensible aux petites perturbations (comportement chaotique).
	 
	 \subsection*{Conséquences forensiques}
	 Une petite falsification (effacement ou modification d'un log) peut conduire à une reconstruction temporelle fortement biaisée. L'investigateur doit :
	 \begin{itemize}
	 	\item introduire des marges d'incertitude dans la timeline,
	 	\item recouper plusieurs sources indépendantes (logs réseaux, SIEM, appliances, sauvegardes),
	 	\item utiliser des méthodes probabilistes et bayésiennes pour estimer la vraisemblance des scénarios.
	 \end{itemize}
	 \section*{Partie 3 : Révolution Quantique et Ses Implications}
	 \subsection*{6. Expérience de Pensée Schrödinger Adaptée }
	 \subsection*{7. Calculs sur la Sphère de Bloch}
	  On considère un qubit avec \(\theta = \pi/3\), \(\varphi = \pi/4\) :
	
	 * Impact en preuve quantique
	 Un système de preuve basé sur qubit ne donnerait pas une certitude absolue mais une probabilité (par ex. 75\% de \(|0\rangle\)). En justice, la preuve deviendrait \emph{probabiliste} plutôt qu'absolue.
	 
	 
	 \subsection*{8.Analyse du Théorème de Non-Clonage }
	
	 \section*{Partie 4 : Paradoxe de l’Authenticité Invisible}
	 
	 \subsection*{ 9: Formalisation mathématique du paradoxe de l'authenticité invisible}
	 
	 Soit une preuve \(P\) caractérisée par :
	 \begin{itemize}
	 	\item Authenticité \(A(P)\),
	 	\item Confidentialité \(C(P)\),
	 	\item Opposabilité \(O(P)\),
	 \end{itemize}
	 tous compris entre 0 et 1.  
	 
	 La relation fondamentale est :
	 \[
	 A(P)\cdot C(P) \leq 1-\delta, \quad \delta > 0.
	 \]
	 
	 *Expérimentation
	 
	 Pour trois systèmes de preuve :
	 \begin{itemize}
	 	\item Système classique : \(A=0.8, C=0.4\),
	 	\item Système blockchain : \(A=0.9, C=0.6\),
	 	\item Système ZK-NR : \(A=0.7, C=0.8\).
	 \end{itemize}
	 
	 On vérifie dans chaque cas l'inégalité.  
	 
	 \textbf{Constante quantique numérique :}
	 Une incertitude analogue à celle d'Heisenberg est postulée :
	 \[
	 \Delta A \cdot \Delta C \geq \frac{\hbar_{\text{num}}}{2}.
	 \]
	 
	 \subsection*{10: Implémentation simplifiée ZK-NR}
	 
	 On peut simuler un protocole ZK-NR en Python :
	 \begin{itemize}
	 	\item Le prouveur \(P\) démontre la possession d'une information sans la révéler.
	 	\item Le vérificateur \(V\) obtient l'assurance mais pas le secret.
	 \end{itemize}
	 
	 \textbf{Évaluation :}
	 \begin{itemize}
	 	\item Confidentialité améliorée,
	 	\item Vérifiabilité maintenue,
	 	\item Coût computationnel supplémentaire (\( \approx 15\%\) de surcharge).
	 \end{itemize}
	 \section*{Partie 5 : Intégration et Synthèse Avancee}
	 
	 \subsection*{11 : Étude de Cas « QuantumLeaks »}
	 
	 \paragraph{Recommandations techniques}
	 \begin{enumerate}
	 	\item \textbf{Chiffrement post-quantique} :
	 	\begin{itemize}
	 		\item Signature : Dilithium ou Falcon
	 		\item Chiffrement : Kyber
	 	\end{itemize}
	 	
	 	\item \textbf{Protocoles de preuve} :
	 	\begin{itemize}
	 		\item ZK-NR pour l'authentification
	 		\item Preuves à divulgation nulle de connaissance
	 	\end{itemize}
	 	
	 	\item \textbf{Archivage à long terme} :
	 	\begin{itemize}
	 		\item Codes correcteurs Reed-Solomon
	 		\item Réplication géo-distribuée
	 	\end{itemize}
	 \end{enumerate}
	 
	 \paragraph{Respect du trilemme CRO}
	 \[
	 \text{CRO} : 
	 \begin{cases}
	 	\text{Confidentialité} \geq 1 - 2^{-128} \\
	 	\text{Fiabibilité} \geq 0.999 \\
	 	\text{Opposabilité} \text{ garantie juridiquement}
	 \end{cases}
	 \]
	 
	 \subsection*{12 : Débat Philosophique Structuré}
	 
	 \paragraph{Thèse réaliste}
	 \begin{itemize}
	 	\item L'observation quantique modifie nécessairement le système
	 	\item La neutralité absolue est physiquement impossible (Wheeler)
	 	\item L'investigateur est un « observateur participant »
	 \end{itemize}
	 
	 \paragraph{Thèse constructiviste}
	 \begin{itemize}
	 	\item La neutralité est un idéal régulateur (Kant)
	 	\item Les pratiques évoluent avec les paradigmes (Kuhn)
	 	\item La réflexivité permet d'approcher la neutralité
	 \end{itemize}
	 
	 \paragraph{Synthèse}
	 L'investigateur post-quantique adopte une position « méta-réflexive » :
	 \begin{itemize}
	 	\item Conscience des biais d'observation
	 	\item Documentation transparente des méthodes
	 	\item Validation intersubjective des résultats
	 \end{itemize}
	 
	 \subsection*{ 13 : Projet de Recherche Personnel}
	 
	
	 
	 \section*{Grille d'Auto-Évaluation}
	  \begin{table}
	 	\begin{tabular}{|l|c|c|c|}
	 		\hline
	 		\textbf{Compétence} & \textbf{Acquis} & \textbf{En cours} & \textbf{À revoir} \\
	 		\hline
	 		Compréhension des fondements philosophiques & & oui & \\
	 		Maîtrise des outils mathématiques & &  & oui \\
	 		Application des concepts quantiques & & oui & \\
	 		Résolution du paradoxe authenticité/confidentialité & oui &  & \\
	 		Intégration interdisciplinaire &  & & oui  \\
	 		\hline
	 	\end{tabular}
	 	\caption{Grille de progression personnelle}
	 \end{table}
	 
	 
\end{document}	
	