\documentclass[12pt,a4paper]{article}
\usepackage[french]{babel}
\usepackage[utf8]{inputenc}
\usepackage[T1]{fontenc}
\usepackage{geometry}
\usepackage{enumitem}
\usepackage{titlesec}
\usepackage{xcolor}
\usepackage{multirow}
\usepackage{booktabs}
\usepackage{tabularx}
\usepackage{graphicx}
\usepackage{float}

\geometry{margin=2.5cm}

% Style des sections
\titleformat{\section}
{\normalfont\Large\bfseries\color{blue!80!black}}
{\thesection}{1em}{}


\begin{document}
	\begin{titlepage}
		\begin{sffamily}
			\begin{center}
				\includegraphics[width=2.5 cm]{poooo.png} % logo centré
			\end{center}
			\begin{center}
				\textsc{\LARGE ÉCOLE NATIONALE SUPÉRIEURE POLYTECHNIQUE DE YAOUNDÉ}\\[2cm]
				\textsc{\Large Département de Génie Informatique}\\[2cm]
				\textsc{\large Introduction aux Techniques d'Investigation Numérique}\\[1.5cm]
			
				{ \huge \bfseries CONSTRUCTION D'HYPOTHESES CONCERNANT LA MORT DE MARTINEZ ZOGO \\[0.4cm] }
				\vfill
				\begin{minipage}{0.4\textwidth}
					\begin{flushleft} \large
						NGUEMO VOUFO AURELLE  \\
						CIN-4\\
					\end{flushleft}
				\end{minipage}
				\begin{minipage}{0.4\textwidth}
					\begin{flushright} \large
						\emph{Superviseur :} M. \textsc{Thierry Minka}\\
					\end{flushright}
				\end{minipage}
				\vfill
			\end{center}
		\end{sffamily}
	\end{titlepage}
	

\tableofcontents
\newpage
	

	
	\section{Introduction}
	Basé sur l'ordonnance de renvoi et les éléments d'enquête disponibles, nous présenterons trois hypothèses sur les circonstances concrètes ayant conduit à la mort de Martinez Zogo survenue le 23 janvier 2023 à SOA. Ces hypothèses s'appuient sur les déclarations des suspects, les preuves techniques et la chronologie des événements.
	
	\section{Hypothèse 1: L'Excès de Violence Non Prémédité}
	
	\subsection{Scénario Principal}
	\begin{itemize}[leftmargin=*]
		\item \textbf{Intention initiale} : Intimidation et correction physique
		\item \textbf{Dérive} : Escalade incontrôlée de la violence
		\item \textbf{Issue fatale} : Conséquence non intentionnelle
	\end{itemize}
	
	
	
	\subsection{Points d'Appui de l'Hypothèse}
	\begin{itemize}[leftmargin=*]
		\item \textbf{Consignes initiales} : "Laisser en vie" répétées par DANWE
		\item \textbf{Déclaration GODJE} : "Nous l'avons laissé bien en vie"
		\item \textbf{Rapport médical} : "Strangulation après torture"
		\item \textbf{Équipement} : Matériel de torture mais pas d'arme à feu
	\end{itemize}
	
	\subsection{Faiblesses de l'Hypothèse}
	\begin{itemize}[leftmargin=*]
		\item Préméditation évidente dans la préparation
		\item Utilisation d'un cutter (arme tranchante)
		\item Deuxième opération coordonnée
	\end{itemize}
	
	\section{Hypothèse 2: L'Assassinat Prémédité et Orchestré}
	
	\subsection{Scénario Principal}
	\begin{itemize}[leftmargin=*]
		\item \textbf{Commande initiale} : Élimination physique déguisée
		\item \textbf{Plan en deux phases} : Torture puis exécution
		\item \textbf{Dénégation plausible} : Séparation des équipes
	\end{itemize}
	
	
	

	\subsection{Preuves de Préméditation}
	\begin{itemize}[leftmargin=*]
		\item \textbf{Financement} : 2 millions FCFA versés d'avance
		\item \textbf{Planification} : Fiches techniques, surveillance prolongée
		\item \textbf{Compartmentalisation} : Deux équipes distinctes
		\item \textbf{Moyens} : Véhicule, armes, équipement spécialisé
		\item \textbf{Alibis préparés} : Permission LAMFU, mission officielle
	\end{itemize}
	
	\subsection{Motifs Cumulatifs}
	\begin{itemize}[leftmargin=*]
		\item \textbf{Règlement de comptes journalistique} : AMOUGOU BELINGA
		\item \textbf{Protection de secrets d'État} : EKO EKO (DGRE)
		\item \textbf{Silence sur corruption} : Documents compromettants
		\item \textbf{Réputation politique} : SAVOM MARTIN
	\end{itemize}
	
	\section{Hypothèse 3: L'Accident Converti en Opportunité}
	
	\subsection{Scénario Principal}
	\begin{itemize}[leftmargin=*]
		\item \textbf{Objectif réel} : Enlèvement et intimidation extrême
		\item \textbf{Accident} : Mort non prévue pendant la torture
		\item \textbf{Adaptation} : Transformation en assassinat couvert
	\end{itemize}
	
	\subsection{Éléments de l'Accident}
	\begin{itemize}[leftmargin=*]
		\item \textbf{Santé fragile} : Zogo asthmatique, faible constitution
		\item \textbf{Violence excessive} : Jeunes militaires peu expérimentés
		\item \textbf{Mauvais timing} : Torture trop longue, saignement non contrôlé
		\item \textbf{Manque de surveillance médicale} : Aucun secouriste présent
	\end{itemize}
	\subsection{Mise en Scène Post-Mortem}
	\begin{itemize}[leftmargin=*]
		\item \textbf{Positionnement} : Corps laissé visible pour découverte rapide
		\item \textbf{Corde au cou} : Suggère un suicide ou assassinat simple
		\item \textbf{Disparition des preuves} : Cutter, câble, huile emportés
		\item \textbf{Coordination des alibis} : Messages post-opération
	\end{itemize}
	
	
	\section{Comparaison des Trois Hypothèses}
	
	\begin{table}[H]
		\centering
		\begin{tabular}{|p{0.22\textwidth}|p{0.25\textwidth}|p{0.25\textwidth}|p{0.25\textwidth}|}
			\hline
			\textbf{Critère} & \textbf{Hypothèse 1: Excès} & \textbf{Hypothèse 2: Prémédité} & \textbf{Hypothèse 3: Accident} \\
			\hline
			\textbf{Intention initiale} & Intimidation & Assassinat & Enlèvement/Torture \\
			\hline
			\textbf{Planification} & Limitée & Élaborée & Modérée \\
			\hline
			\textbf{Responsabilité} & Collective & Hiérarchique & Accidentelle \\
			\hline
			\textbf{Cover-up} & Improvisé & Intégré au plan & Réactif \\
			\hline
			\textbf{Cohérence interne} & Moyenne & Élevée & Élevée \\
			\hline
			\textbf{Risque pénal} & Réduit (homicide) & Maximal (assassinat) & Intermédiaire \\
			\hline
			\textbf{Probabilité} & 30\% & 45\% & 25\% \\
			\hline
		\end{tabular}
		\caption{Analyse comparative des hypothèses}
	\end{table}
	
	\subsection{Points Communs aux Trois Hypothèses}
	\begin{itemize}[leftmargin=*]
		\item \textbf{Enlèvement planifié} : Sur la base de renseignements DGRE
		\item \textbf{Torture systématique} : Méthode organisée et brutale
		\item \textbf{Deuxième intervention} : Équipe séparée pour l'étape finale
		\item \textbf{Motif de silenciation} : Empêcher Zogo de révéler des informations
	\end{itemize}
	
	\subsection{Éléments en Faveur de la Préméditation}
	\begin{itemize}[leftmargin=*]
		\item \textbf{Financement antérieur} : 2 millions avant l'opération
		\item \textbf{Documentation préparée} : Fiches technique et géolocalisation
		\item \textbf{Compartmentalisation} : Équipes distinctes avec rôles précis
		\item \textbf{Chronologie serrée} : Fenêtre temporelle précise
		\item \textbf{Post-opération} : Paiements aux participants
	\end{itemize}
	
	\section{Conclusion}
	
	L'analyse des trois hypothèses suggère que la vérité se situe probablement entre l'hypothèse 2 (assassinat prémédité) et l'hypothèse 3 (accident converti en opportunité). 	\textbf{Hypothèse la plus probable} : Assassinat prémédité sous couvert d'opération "correction", avec dépassement possible des violences initialement prévues, transformant une mission d'intimidation violente en exécution fatale.
	
\end{document}